\section{Introduction}

Advanced extracellular recording technologies now enable neuroscientists to capture unprecedented volumes of high-dimensional neural data at single-cell resolution ~\cite{steinmetz_2021_neuropixels2, raducanu_2017_neuroseeker, cai_2016_miniscope, villette_2019_voltage_2p, ouzounov_2017_three_photon, ahrens_2013_lightsheet}. Understanding the computational principles driving neural activity requires extracting interpretable features from this data. Traditional dimensionality reduction techniques, while useful for visualization and basic analyses, often fail to disentangle distributed representations to provide a mechanistic understanding of the underlying computations. ~\cite{cunningham_2014_neural_dr, humphries_2021_dr_principles, hotelling_1933_pca, hoyer_2004_sparsenmf}. Recent work has produced promising latent variable model (LVM) approaches capable of identifying low-dimensional subspaces that can accurately decode aspects of behavior and environment ~\cite{song_2025_langevinflow, schneider_2023_cebra, le_2022_stndt, keshtkaran_2022_autolfads, yu_2009_gpfa, macke_2011_plds, gao_2016_pflds, low_2018_mind, jensen_2020_mgplvm, hernandez_2020_vind, kim_2021_plnde, hurwitz_2021_tndm, schimel_2022_ilqrvae, kudryashova_2023_ctrltndm, ye_2023_ndt2, gondur_2024_mmgpvae, pellegrino_2024_slicetca, sani_2024_dpad, pals_2024_smclr_rnn, zhang_2024_mtm, george_2025_simpl, perkins_2025_mint, schmutz_2025_nce}. However, these methods typically value decoding accuracy over latent interpretability, and consequently all have at least one of the following limitations: poorly interpretable latents, lossy mapping to and/or poor reconstruction of neural data from latent space, priors on the latent and/or neural data space, requirements of additional behavioral, environmental, and/or trial-structured neural data, supralinear scaling with respect to dataset size, or high implementational complexity.

To overcome these limitations, we introduce the Mechanistic Interpretability for Neural Interpretability (MINI) pipeline. MINI provides a \textit{simple, user-friendly library} for \textit{interpretable latent discovery} from high-dimensional neural data. We consider a latent's interpretability in two key aspects: 1) its correspondence to a specific external variable -- a "natural", behavioral or environmental feature; 2) its explicit composition from contributing neural activity. MINI's approach obviates the need for post-hoc analyses to find meaningful dynamics in latent spaces typical of other approaches. Additionally, while we apply MINI to neural spike data in this work, it can be readily applied to virtually any other neural recording modality as the only neural data requirement is any predefined spatiotemporal binning.

Conceptually, MINI is motivated by the principle that we need not explicitly model the entire latent space nor its dynamics to discover interpretable variables. Specifically, MINI bypasses the need to enforce non-Markovian dynamics, which are often a property of an incomplete observation of a system rather than the system itself; any dynamical system can be represented as Markovian via sufficient state augmentation ~\cite{takens_1981_embedding}. From this viewpoint, even methods that appear non-Markovian can be seen as attempts to approximate this complete underlying state. For example, the forward dynamics of a recurrent neural network (RNN) are Markovian in its hidden state ~\cite{sussillo_2013_rnn_dynamics, goodfellow_2016_rnn}, and generalized linear models (GLMs) with history filters ~\cite{pillow_2008_glms, truccolo_2005_pointprocess} use past activity to augment the present state. This perspective aligns with several findings in neuroscience, from activity in motor cortex reflecting a low-dimensional, implicitly Markovian dynamical system ~\cite{churchland_2012_population_dynamics}, to predictive-coding formulations that posit that the brain maintains an internal state sufficient to predict sensory streams, i.e., a Markovian latent process \cite{rao_1999_predictive_coding, doya_2007_bayesian_brain, friston_2010_free_energy}. Grounded in this perspective, MINI forgoes the challenge of modeling intricate dynamics and instead directly extracts meaningful constituent features of the underlying neural state.

% ---


\begin{enumerate}
\item \textbf{Multi-scale temporal dynamics}: Neural computations involve fast synaptic events, intermediate-timescale neural oscillations, and slow behavioral modulations that require different temporal resolutions for optimal feature extraction.

\item \textbf{Interpretability vs. reconstruction trade-off}: While sparsity promotes interpretability, it must be balanced with the ability to accurately reconstruct the original neural signals for downstream analyses.

\item \textbf{Scalability to large datasets}: Modern neural datasets contain millions of recorded time points across hundreds to thousands of neurons, requiring efficient algorithms that can scale to these data volumes.

\item \textbf{Cross-dataset generalization}: Features discovered in one experimental context should generalize to related experimental paradigms and neural systems.
\end{enumerate}

\subsection{Related Work}

% TODO: Expand this section with specific citations and comparisons
Recent advances in interpretable machine learning have focused on developing methods that provide both predictive accuracy and mechanistic understanding. In neuroscience, several approaches have been developed for extracting interpretable features from neural data:

\textbf{Traditional dimensionality reduction}: Principal Component Analysis (PCA) and Independent Component Analysis (ICA) have been widely used but often produce features that lack clear biological interpretation.

\textbf{Sparse coding approaches}: Methods such as non-negative matrix factorization and dictionary learning have shown promise for identifying localized neural patterns.

\textbf{Deep learning for neuroscience}: Recent work has applied various deep learning architectures to neural data, including variational autoencoders, recurrent neural networks, and transformers.

\textbf{Sparse autoencoders}: SAEs have gained attention for their ability to learn interpretable features while maintaining reconstruction quality, with applications ranging from computer vision to neuroscience.

Our work builds upon these foundations by specifically addressing the multi-scale nature of neural computations and providing systematic evaluation across multiple large-scale neural datasets.

% TODO: Add specific contributions and outline of the paper
\subsection{Contributions}

This paper makes the following key contributions:

\begin{enumerate}
\item We introduce Multi-Scale Sparse Autoencoders (MSAEs) specifically designed for neural data analysis, extending traditional SAEs to capture multi-scale temporal and spatial patterns.

\item We develop a comprehensive pipeline for neural data preprocessing, model training, evaluation, and feature interpretation that can be applied across different experimental paradigms.

\item We provide systematic evaluation on multiple large-scale neural datasets, demonstrating the effectiveness of our approach across different brain regions, recording techniques, and experimental conditions.

\item We show that MSAE-derived features enable effective decoding of behavioral and stimulus variables, validating their biological relevance.

\item We release open-source implementations and interactive tools for feature exploration, facilitating broader adoption in the neuroscience community.
\end{enumerate}
