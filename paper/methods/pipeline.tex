\subsection{Pipeline Overview}

- Overall: MINI takes in high-dimensional neural data and outputs latents to evaluate

- For a user, the full, semi-automated pipeline is as follows:
\begin{enumerate}
    \item Data preprocessing
    \begin{itemize}
        \item Spatiotemporally bin and normalize
        \begin{itemize}
            \item MINI has a convenience function to do this directly from output of common spikesorters (kilosort), where we bin unit spikes given a specified timebin and optionally normalize (z-score or max) dataset across time and/or unit
            \begin{itemize}
                \item (and similar approach could be applied to output from common calcium imaging processing (e.g. Suite2p))
            \end{itemize}
        \end{itemize}
    \end{itemize}
    
    \item Model training
    \begin{itemize}
        \item Hyperparameter optimization
        \item By default no validation set, but can be added if we want to e.g. apply to other recordings of same animal, though this is not generally recommended (just train a freshie)
    \end{itemize}
    
    \item Model evaluation
    \begin{itemize}
        \item (We implement all metrics from SAEBench which are not language-model specific, plus a couple of our own)
        \item L0 of latents
        \item R\textsuperscript{2} (var explained) and cos sim of reconstruction-to-actual neural activity for each spatial bin, and each temporal bin
        \item Latent density histogram (as in SAEBench)
        \item Variance explained of overall reconstruction from each latent (variance shouldn't be in just a few features) ?
        \item Spectral frequency analysis to ensure temporal frequency content is preserved?
    \end{itemize}
    
    \item Feature evaluation
    \begin{itemize}
        \item When a latent is deemed sufficiently interpretable, we call this a feature.
        \item Interactive plots showing feature activation patterns across time and experimental conditions.
        \item We evaluate its decoding performance?
        \item Export functionality.
    \end{itemize}
\end{enumerate}
