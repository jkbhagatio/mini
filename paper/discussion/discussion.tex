\section{Discussion}

% TODO: Edit
\subsection{Contributions}

This paper makes the following key contributions:

\begin{enumerate}
\item We introduce Multi-Scale Sparse Autoencoders (MSAEs) specifically designed for neural data analysis, extending traditional SAEs to capture multi-scale temporal and spatial patterns.

\item We develop a comprehensive pipeline for neural data preprocessing, model training, evaluation, and feature interpretation that can be applied across different experimental paradigms.

\item We provide systematic evaluation on multiple large-scale neural datasets, demonstrating the effectiveness of our approach across different brain regions, recording techniques, and experimental conditions.

\item We show that MSAE-derived features enable effective decoding of behavioral and stimulus variables, validating their biological relevance.

\item We release open-source implementations and interactive tools for feature exploration, facilitating broader adoption in the neuroscience community.
\end{enumerate}

\subsection{Summary}

In this work, we introduced a multi-scale sparse autoencoder (MSAE) for discovering interpretable neural features from large-scale electrophysiology datasets. Our approach effectively captures neural dynamics across multiple temporal resolutions, revealing features that correspond to known and novel aspects of neural computation.

\subsection{Pros and Cons}

\textbf{Advantages}:
\begin{itemize}
    \item \textbf{Interpretability}: The sparse and multi-scale nature of the learned features facilitates direct interpretation and comparison with existing neuroscience knowledge.
    \item \textbf{Scalability}: The model architecture is designed to scale to large datasets with thousands of neurons and millions of time points.
    \item \textbf{Flexibility}: The framework can be adapted to various data modalities, including spike trains, calcium imaging, and LFP/EEG signals.
\end{itemize}

\textbf{Limitations}:
\begin{itemize}
    \item \textbf{Hyperparameter sensitivity}: Model performance is sensitive to the choice of hyperparameters, requiring careful tuning and validation.
    \item \textbf{Causality}: While our model reveals correlational structure, it does not inherently infer causal relationships between neural features and behavior.
    \item \textbf{Biological realism}: The model is a simplified representation of neural circuits and does not capture all aspects of biological complexity.
\end{itemize}

\subsection{Future Work}

\subsubsection{Extend to other recording modalities}
Our framework can be extended to other neural recording modalities. For LFP and EEG data, the model can be adapted to reconstruct voltage values across channels over time. For calcium imaging, it can reconstruct dF/F signals from regions of interest (ROIs).

\subsubsection{Spatio-Temporal Convolutional Components (SCCs)}
Future work will explore the use of spatio-temporal convolutional components (SCCs) to model neural dynamics more explicitly. This could enable applications such as:
\begin{itemize}
    \item \textbf{Brain diffing}: Identifying differences in neural dynamics across experimental conditions or subject groups.
    \item \textbf{Cross-region prediction}: Predicting neural activity in one brain region based on activity in another.
\end{itemize}

\subsubsection{Sequence Length and History Dependence}
Investigating the role of sequence length and history dependence could improve reconstruction accuracy and reveal features that depend on longer time scales. Recurrent or temporal convolutional network components could be integrated into the model to capture these dependencies.
