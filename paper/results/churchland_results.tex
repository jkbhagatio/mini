\subsection{Natural macaque spike data in a reaching task}

Here we compare MINI to other popular methods that have also been applied to this dataset.

To do a fair comparison with MINI's main goal of unsupervised interpretable latent discovery, we detail comparisons (\nameref{table:method_comparisons}) with paper-published methods that:

\begin{enumerate}
\item Claim an interpretable latent space among their primary goals (as opposed to e.g. only strong decoding performance)
\item Don't require multimodal nor trial-structured data
\end{enumerate}
    
Among these, we visualize results from LangevinFlow and CEBRA, as they represent current neural latents benchmark[ref] state-of-the-art methods for an autoencoder approach and non autoencoder approach, respectively. 

We also include sparseNMF[ref], t-SNE[ref], and UMAP[ref] as baselines, as they are widely used general methods for dimensionality reduction and latent extraction.
(We don't show ICA, as (ICA < CEBRA) nor PCA, as (PCA < sparseNMF) ).

Brief dataset info...
\begin{itemize}
    \item Task info \& metadata
    \item Brain regions \& units
\end{itemize}

Features found and decoding accuracy comparisons with LangevinFlow and CEBRA...
\begin{enumerate}
    \item Direction
    \item Velocity
    \item Movement onset
    \item Trial type
    \item Features across different timescales (i.e. that persevere for different durations)
    \item Communication across brain regions??
\end{enumerate}

% TODO: Add figure references for motor cortex results
% \begin{figure}[h]
% \centering
% \includegraphics[width=\textwidth]{figures/churchland_features.pdf}
% \caption{Motor cortical features discovered by MSAEs.}
% \label{fig:churchland_features}
% \end{figure}
