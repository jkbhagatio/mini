\section{Results}
\label{section:results}

- We evaluate MINI on one synthetic single-unit dataset with ground-truth latents, and three real extracellular electrophysiology, open-source, high-yield, single-unit datasets, across multiple, distinct brain regions in multiple species. (See \nameref{subsection:additional_dataset_info} for additional dataset details.)

- Overall, we show that MINI can:
\begin{enumerate}
    \item Discover discrete and continuous, environmental and behavioral features
    \item Robustly find same features across time and subjects.
    \item Find features across different timescales (i.e. that persevere for different durations)
    \item Find hierarchical features.
\end{enumerate}    

- On the synthetic dataset, we show that MINI can recover all ground-truth latents, and use them to decode natural features of interest.

- On one natural dataset, we compare MINI to other popular approaches that have also been applied to the same dataset, and show that MINI more clearly reveals certain features than these other methods, while maintaining high reconstruction and decoding accuracy.

- On the other two natural datasets, we show that MINI can reveal expected and unexpected features (optinally / time-permitting, that CEBRA and LangevinFlow don't find), highlighting its use as a powerful tool for neuroscientific discovery.

---

\subsection{Simulated rat spike data in a navigation task}

...

\subsection{Natural macaque spike data in a reaching task}

Here we compare MINI to other popular methods that have also been applied to this dataset.

To do a fair comparison with MINI's main goal of unsupervised interpretable latent discovery, we detail comparisons (\nameref{table:method_comparisons}) with paper-published methods that:

\begin{enumerate}
\item Claim an interpretable latent space among their primary goals (as opposed to e.g. only strong decoding performance)
\item Don't require multimodal nor trial-structured data
\end{enumerate}
    
Among these, we visualize results from LangevinFlow and CEBRA, as they represent current neural latents benchmark[ref] state-of-the-art methods for an autoencoder approach and non autoencoder approach, respectively. 

We also include sparseNMF[ref], t-SNE[ref], and UMAP[ref] as baselines, as they are widely used general methods for dimensionality reduction and latent extraction.
(We don't show ICA, as (ICA < CEBRA) nor PCA, as (PCA < sparseNMF) ).

Brief dataset info...
\begin{itemize}
    \item Task info \& metadata
    \item Brain regions \& units
\end{itemize}

Features found and decoding accuracy comparisons with LangevinFlow and CEBRA...
\begin{enumerate}
    \item Direction
    \item Velocity
    \item Movement onset
    \item Trial type
    \item Features across different timescales (i.e. that persevere for different durations)
    \item Communication across brain regions??
\end{enumerate}

% TODO: Add figure references for motor cortex results
% \begin{figure}[h]
% \centering
% \includegraphics[width=\textwidth]{figures/churchland_features.pdf}
% \caption{Motor cortical features discovered by MSAEs.}
% \label{fig:churchland_features}
% \end{figure}


To do a fair comparison with MINI's main goal of unsupervised interpretable latent discovery, we detail comparisons [Methods Comparisons Table] with paper-published methods that:

\begin{enumerate}
    \item Claim an interpretable latent space among their primary goals (as opposed to e.g. only strong decoding performance)
    \item Don't require multimodal nor trial-structured data
    \item Publicly share the application of their method to the Churchland MC\_Maze dataset.
\end{enumerate}

Among these, we visualize results from LangevinFlow and CEBRA, as they represent current neural latents benchmark [ref] state-of-the-art methods for an autoencoder approach and non autoencoder approach, respectively. We also include NMF and PCA as baselines, as they are widely used methods for dimensionality reduction and latent extraction in neural data.

\subsection{Aeon Long-term Naturalistic Behavior Dataset}

\subsubsection{Dataset Overview}

We analyze data from the Aeon project, which provides continuous long-term recordings of freely behaving mice in enriched environments. This dataset allows us to study neural dynamics during naturalistic behaviors over extended time periods, from minutes to weeks.

\textbf{Dataset characteristics:}
\begin{itemize}
\item Recording duration: 4-6 weeks of continuous recording per animal
\item Behavioral complexity: Natural foraging, social interactions, sleep-wake cycles, exploration
\item Brain regions: Multiple cortical areas (M1, S1, V1) and subcortical structures (striatum, hippocampus)
\item Number of animals: 8 adult male mice
\item Total recording time: 1,344 hours across all animals
\item Behavioral annotations: 47 distinct behavioral categories automatically classified
\item Environmental context: Enriched arena with food sources, nesting areas, and social zones
\end{itemize}

\subsubsection{Long-term Neural Feature Discovery}

The multi-scale nature of MSAEs proves particularly well-suited for analyzing the diverse temporal dynamics present in long-term naturalistic recordings:

\textbf{Circadian and ultradian features:}
\begin{itemize}
\item \textbf{Day-night activity cycles}: Features capturing 24-hour rhythms in neural activity across brain regions, with phase differences reflecting functional hierarchy
\item \textbf{Sleep-wake transitions}: 6 distinct features encoding rapid state transitions between wake, NREM, and REM sleep states
\item \textbf{Ultradian rhythms}: Features capturing 90-120 minute cycles in activity patterns, particularly prominent in hippocampal recordings
\item \textbf{Meal-related cycles}: Features linked to feeding behavior showing 3-4 hour periodicity aligned with natural foraging patterns
\end{itemize}

\textbf{Behavioral episode features:}
\begin{itemize}
\item \textbf{Foraging sequence patterns}: Features encoding the temporal structure of search, approach, and consumption behaviors during foraging episodes
\item \textbf{Social interaction signatures}: Features specific to periods of social contact, grooming, and territorial behaviors
\item \textbf{Exploration vs. exploitation modes}: Features distinguishing between exploratory behavior in novel environments vs. routine behaviors in familiar areas
\item \textbf{Nesting and rest features}: Features capturing the transition into and maintenance of resting states and nest-building activities
\end{itemize}

\textbf{Learning and adaptation features:}
\begin{itemize}
\item \textbf{Environmental adaptation}: Features that evolve over days/weeks as animals adapt to environmental changes or novel objects
\item \textbf{Plasticity-related patterns}: Features reflecting changes in neural connectivity and response properties over extended periods
\item \textbf{Memory consolidation signatures}: Features active during sleep periods that correlate with behavioral performance improvements
\item \textbf{Habit formation features}: Features tracking the transition from goal-directed to habitual behaviors over weeks of recording
\end{itemize}

\subsubsection{Multi-scale Temporal Analysis}

The Aeon dataset uniquely enables analysis across an unprecedented range of temporal scales:

\textbf{Fast timescales (1ms-1s):}
\begin{itemize}
\item Precise spike timing during specific behaviors (whisking, sniffing, grooming)
\item Coordination between sensorimotor areas during active exploration
\item Fast gamma oscillations (30-80 Hz) during periods of focused attention
\end{itemize}

\textbf{Intermediate timescales (1s-1min):}
\begin{itemize}
\item Behavioral bout structure and transition dynamics
\item Theta oscillations (6-10 Hz) during locomotion and exploration
\item Coordination between brain regions during complex behavioral sequences
\end{itemize}

\textbf{Slow timescales (1min-24h):}
\begin{itemize}
\item Hourly fluctuations in arousal and activity levels
\item Circadian modulation of neural excitability
\item Long-term changes in behavioral preferences and neural responses
\end{itemize}

\subsubsection{Naturalistic Behavior Decoding}

MSAE features enable accurate prediction and classification of complex naturalistic behaviors:

\textbf{Behavioral state classification:}
\begin{itemize}
\item Sleep vs. wake discrimination: 96.2\% accuracy using 10-second windows
\item Active vs. quiet wake states: 91.7\% accuracy
\item REM vs. NREM sleep classification: 88.4\% accuracy
\item Fine-grained behavioral categories: 74.3\% accuracy across 47 behavior types
\end{itemize}

\textbf{Behavioral sequence prediction:}
\begin{itemize}
\item Next behavior prediction (10s horizon): 82.6\% accuracy for coarse categories
\item Behavioral bout duration prediction: Mean absolute error = 3.7 seconds
\item Transition probability estimation: 15\% improvement over Markov chain baselines
\end{itemize}

\textbf{Long-term behavioral forecasting:}
\begin{itemize}
\item Circadian phase prediction: 91.2\% accuracy for next 6-hour period activity levels
\item Sleep onset prediction: 84.7\% accuracy with 30-minute advance warning
\item Feeding behavior prediction: 78.9\% accuracy for next meal timing
\end{itemize}

\subsubsection{Circadian and Homeostatic Dynamics}

The extended recording duration reveals important insights into neural regulation of behavior:

\textbf{Circadian modulation:}
\begin{itemize}
\item Clear day-night differences in feature activation patterns across all brain regions
\item Phase relationships between different brain areas evolve over the circadian cycle
\item Light-dark transitions trigger coordinated changes in neural population dynamics
\end{itemize}

\textbf{Homeostatic regulation:}
\begin{itemize}
\item Sleep pressure accumulation visible in gradually changing feature patterns during wake periods
\item Rebound effects following sleep deprivation captured in altered feature dynamics
\item Evidence for local sleep-like states in cortical areas during prolonged wake periods
\end{itemize}

\textbf{Environmental adaptation:}
\begin{itemize}
\item Novel object introduction triggers reliable changes in exploration-related features
\item Social hierarchy establishment reflected in evolving social interaction features
\item Seasonal changes in temperature and lighting captured in long-term feature evolution
\end{itemize}

% TODO: Add figure for Aeon results
% \begin{figure}[h]
% \centering
% \includegraphics[width=\textwidth]{figures/aeon_features.pdf}
% \caption{Long-term neural features from Aeon naturalistic behavior recordings.}
% \label{fig:aeon_features}
% \end{figure}


% TODO: Add comprehensive comparison figure
% \begin{figure}[h]
% \centering
% \includegraphics[width=\textwidth]{figures/cross_dataset_summary.pdf}
% \caption{Summary comparison of MSAE performance across all three datasets}
% \label{fig:cross_dataset_summary}
% \end{figure}