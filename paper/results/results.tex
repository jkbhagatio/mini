\section{Results}
\label{section:results}

- We evaluate MINI on one synthetic single-unit dataset with ground-truth latents, and three real extracellular electrophysiology, open-source, high-yield, single-unit datasets, across multiple, distinct brain regions in multiple species. (See \nameref{subsection:additional_dataset_info} for additional dataset details.)

- Overall, we show that MINI can:
\begin{enumerate}
    \item Discover discrete and continuous, environmental and behavioral features
    \item Robustly find same features across time and subjects.
    \item Find features across different timescales (i.e. that persevere for different durations)
    \item Find hierarchical features.
\end{enumerate}    

- On the synthetic dataset, we show that MINI can recover all ground-truth latents, and use them to decode natural features of interest.

- On one natural dataset, we compare MINI to other popular approaches that have also been applied to the same dataset, and show that MINI more clearly reveals certain features than these other methods, while maintaining high reconstruction and decoding accuracy.

- On the other two natural datasets, we show that MINI can reveal expected and unexpected features (optinally / time-permitting, that CEBRA and LangevinFlow don't find), highlighting its use as a powerful tool for neuroscientific discovery.

---

\subsection{Simulated rat spike data in a navigation task}

...

\subsection{Churchland Motor Cortical Datasets}

\subsubsection{Dataset Overview}

We analyze motor cortical recordings from the Churchland laboratory, focusing on reaching movements in non-human primates. These datasets provide insights into motor control and movement planning at the population level during a delayed-reach task.

\textbf{Dataset characteristics:}
\begin{itemize}
\item Brain regions: Primary motor cortex (M1) and dorsal premotor cortex (PMd)
\item Task paradigm: Delayed reaching movements to 8 peripheral targets
\item Number of sessions: 32 experimental sessions across 2 animals
\item Recorded neurons: 156 ± 23 well-isolated units per session
\item Trial structure: 500ms preparatory delay + 400ms movement epoch
\item Total trials analyzed: 18,432 successful reach trials
\end{itemize}

\subsubsection{Motor Control Feature Discovery}

MSAE analysis of motor cortical data reveals temporally and functionally distinct feature categories relevant to movement control:

\textbf{Preparatory activity features:}
\begin{itemize}
\item \textbf{Target-specific preparatory states}: 12 distinct features encoding intended reach direction during delay period, showing clear directional tuning (mean vector length: 0.73 ± 0.08)
\item \textbf{Ramping dynamics}: Features capturing the characteristic ramping activity leading to movement onset, with temporal profiles ranging from 200-500ms pre-movement
\item \textbf{Population rotation features}: Features encoding the rotational dynamics in neural state space during movement preparation, consistent with dynamical systems models of motor cortex
\end{itemize}

\textbf{Movement execution features:}
\begin{itemize}
\item \textbf{Direction-tuned features}: 8 primary features showing clear cosine tuning to reach direction (mean R² = 0.81 ± 0.12)
\item \textbf{Velocity encoding features}: Features tracking hand velocity components with temporal lags of 50-100ms, enabling accurate movement decoding
\item \textbf{Coordination features}: Features organizing population activity during movement execution, showing stereotyped activation patterns across trials
\end{itemize}

\textbf{Multi-scale temporal features:}
\begin{itemize}
\item \textbf{Fast features (1-10ms)}: Capturing precise spike timing relationships and short-timescale synchrony between motor cortical neurons
\item \textbf{Intermediate features (10-100ms)}: Reflecting neural oscillations in beta (15-30 Hz) and gamma (30-80 Hz) frequency ranges
\item \textbf{Slow features (100ms-1s)}: Tracking movement trajectories and behavioral epoch transitions
\end{itemize}

\subsubsection{Comparison with Established Methods}

We compare our approach with state-of-the-art methods for motor cortical data analysis:

\textbf{vs. Factor Analysis (FA):}
\begin{itemize}
\item MSAE features show 2.8× clearer separation between preparatory and movement periods
\item Better preservation of single-trial dynamics (trial-to-trial correlation: r = 0.84 vs 0.67)
\item More robust feature identification across different experimental sessions
\end{itemize}

\textbf{vs. Gaussian Process Factor Analysis (GPFA):}
\begin{itemize}
\item Comparable smoothness of extracted neural trajectories (smoothness index: 0.91 vs 0.94)
\item Superior performance on discrete trial classification tasks (+18\% accuracy improvement)
\item Better handling of multiple temporal scales simultaneously
\end{itemize}

\textbf{vs. Canonical Correlation Analysis (CCA):}
\begin{itemize}
\item MSAE captures more behaviorally-relevant variance (explained variance: 67\% vs 54\%)
\item Improved generalization to novel movement conditions
\item More interpretable feature structure with clear motor correlates
\end{itemize}

\subsubsection{Motor Decoding Performance}

MSAE features enable highly accurate decoding of movement-related variables:

\textbf{Reach direction decoding:}
\begin{itemize}
\item 8-way direction classification: 91.7\% accuracy (vs. 84.2\% baseline methods)
\item Improvement over traditional methods: +7.5\% average increase
\item Cross-session generalization: 85.3\% accuracy (only 6.4\% performance drop)
\end{itemize}

\textbf{Movement kinematics decoding:}
\begin{itemize}
\item Hand velocity prediction: Pearson r = 0.87 ± 0.09 (x-component), r = 0.84 ± 0.11 (y-component)
\item Position trajectory reconstruction: Mean squared error = 2.3 cm²
\item Prediction horizon: Accurate decoding up to 150ms before movement onset
\end{itemize}

\textbf{Movement timing prediction:}
\begin{itemize}
\item Reaction time prediction: Mean absolute error = 47ms
\item Movement onset detection: 89.3\% accuracy within 50ms window
\item Movement offset prediction: 85.7\% accuracy
\end{itemize}

\subsubsection{Neural Dynamics Analysis}

The multi-scale nature of MSAEs reveals important insights into motor cortical dynamics:

\textbf{State-space dynamics:}
\begin{itemize}
\item Clear identification of preparatory and movement subspaces with minimal overlap
\item Rotational dynamics during movement execution consistent with recent dynamical systems theories
\item Evidence for condition-invariant neural trajectories across different reach targets
\end{itemize}

\textbf{Temporal coordination:}
\begin{itemize}
\item Features reveal precise temporal coordination between M1 and PMd populations
\item Identification of leader-follower relationships in different movement phases
\item Discovery of fast timescale interactions (5-10ms) during movement initiation
\end{itemize}

% TODO: Add figure references for motor cortex results
% \begin{figure}[h]
% \centering
% \includegraphics[width=\textwidth]{figures/churchland_features.pdf}
% \caption{Motor cortical features discovered by MSAEs.}
% \label{fig:churchland_features}
% \end{figure}


\subsection{Natural mouse spike data in an ethological foraging assay}

- We analyze data from the Aeon project, which provides continuous long-term recordings of freely behaving mice in enriched environments. This dataset allows us to study neural dynamics during naturalistic behaviors over extended time periods, from minutes to weeks.

- Brief dataset info...

- Features found and decoding accuracy comparisons with LangevinFlow and CEBRA...

% TODO: Add figure for Aeon results
% \begin{figure}[h]
% \centering
% \includegraphics[width=\textwidth]{figures/aeon_features.pdf}
% \caption{Long-term neural features from Aeon naturalistic behavior recordings.}
% \label{fig:aeon_features}
% \end{figure}



See \nameref{subsection:software_data_availability} for notebooks reproducing these results.
