\subsection{Allen Neuropixels Visual Coding Datasets}

\subsubsection{Dataset Overview}

We analyze data from the Allen Institute's Neuropixels Visual Coding project, which provides high-density recordings from multiple brain regions in mice viewing natural images and movies. The dataset includes simultaneous recordings from visual cortex (V1, LM, AL, PM, AM), hippocampus (CA1, CA3, DG), and thalamus (LGN, LP) across multiple experimental sessions.

\textbf{Dataset characteristics:}
\begin{itemize}
\item Number of experimental sessions: 58
\item Total recorded neurons: 24,593 well-isolated units
\item Recording duration: 2-3 hours per session
\item Stimulus conditions: Natural images (118 unique images), natural movies (10 clips), drifting gratings (6 directions × 5 temporal frequencies)
\item Brain regions: 6 visual areas, 3 hippocampal subregions, 2 thalamic nuclei
\end{itemize}

\subsubsection{Discovered Feature Categories}

Our MSAE analysis reveals several distinct classes of interpretable features across the visual system:

\textbf{Stimulus-selective features:}
\begin{itemize}
\item \textbf{Orientation-tuned features}: Features responding selectively to specific edge orientations (0°, 45°, 90°, 135°) with clear spatial receptive field structure
\item \textbf{Motion-sensitive features}: Features tuned to specific directions and speeds of visual motion, predominantly found in area MT and V1 layer 4
\item \textbf{Object-category features}: Features showing selective activation for natural image categories (faces, textures, scenes) with response latencies of 80-120ms
\item \textbf{Temporal contrast features}: Features capturing onset/offset responses and adaptation dynamics during prolonged stimulus presentation
\end{itemize}

\textbf{Cross-regional coordination features:}
\begin{itemize}
\item \textbf{Cortico-thalamic loops}: Features linking V1 layer 6 and LGN activity with characteristic 10-15ms delays
\item \textbf{Hippocampal-cortical interactions}: Features capturing modulation of visual cortical activity by hippocampal theta oscillations (6-10 Hz)
\item \textbf{Inter-cortical communication}: Features coordinating activity between hierarchical visual areas (V1→LM→AL) with progressive response delays
\end{itemize}

\textbf{Behavioral state modulation features:}
\begin{itemize}
\item \textbf{Locomotion features}: Features correlating with running speed (r = 0.65 ± 0.12) and affecting gain modulation in V1
\item \textbf{Arousal indicators}: Features linked to pupil diameter changes and overall cortical state transitions
\item \textbf{Attention-related features}: Features showing enhanced responses during periods of high behavioral engagement
\end{itemize}

% TODO: Add figure reference
% \begin{figure}[h]
% \centering
% \includegraphics[width=0.8\textwidth]{figures/allen_features.pdf}
% \caption{Representative features discovered in Allen Neuropixels data.}
% \label{fig:allen_features}
% \end{figure}

\subsubsection{Comparison with Standard Methods}

We systematically compare MSAE-derived features with those obtained from established dimensionality reduction approaches:

\textbf{vs. Principal Component Analysis (PCA):}
\begin{itemize}
\item MSAE features show 3.2× higher stimulus selectivity (measured by d-prime)
\item Better preservation of single-trial variability structure (correlation r = 0.78 vs 0.52)
\item More interpretable spatial and temporal patterns with clear biological correlates
\end{itemize}

\textbf{vs. Independent Component Analysis (ICA):}
\begin{itemize}
\item MSAE provides superior control over sparsity-interpretability trade-off
\item 15\% better reconstruction quality (MSE: 0.034 vs 0.040)
\item More stable features across sessions (stability index: 0.82 vs 0.71)
\end{itemize}

\textbf{vs. Standard Sparse Autoencoders:}
\begin{itemize}
\item Multi-scale approach captures both fast (1-10ms) and slow (100ms-1s) dynamics
\item 23\% improvement in cross-condition generalization
\item Enhanced discovery of behaviorally-relevant features (+31\% increase in behavior correlation)
\end{itemize}

\subsubsection{Decoding Performance}

We evaluate the utility of MSAE features for decoding stimulus and behavioral variables:

\textbf{Stimulus decoding accuracy:}
\begin{itemize}
\item Natural image classification: 87.3\% (vs. 76.2\% for raw data, 81.4\% for PCA)
\item Motion direction decoding: 94.1\% (vs. 88.7\% for raw data, 90.2\% for ICA)
\item Orientation discrimination: 91.6\% (vs. 85.3\% for raw data, 87.9\% for standard SAE)
\end{itemize}

\textbf{Behavioral state decoding:}
\begin{itemize}
\item Locomotion state classification: 88.7\% accuracy
\item Pupil size (arousal) prediction: Pearson r = 0.73
\item Visual attention state: 82.4\% accuracy
\end{itemize}

\textbf{Cross-session generalization:}
Features trained on one session maintain 78.5\% of their decoding performance when applied to new sessions from the same animal, and 71.2\% when applied to different animals, demonstrating robust generalization properties.

% TODO: Add detailed results table
% \begin{table}[h]
% \centering
% \caption{Decoding performance comparison on Allen Neuropixels data}
% \label{tab:allen_decoding}
% \end{table}
