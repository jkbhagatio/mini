\subsection{Aeon Long-term Naturalistic Behavior Dataset}

\subsubsection{Dataset Overview}

We analyze data from the Aeon project, which provides continuous long-term recordings of freely behaving mice in enriched environments. This dataset allows us to study neural dynamics during naturalistic behaviors over extended time periods, from minutes to weeks.

\textbf{Dataset characteristics:}
\begin{itemize}
\item Recording duration: 4-6 weeks of continuous recording per animal
\item Behavioral complexity: Natural foraging, social interactions, sleep-wake cycles, exploration
\item Brain regions: Multiple cortical areas (M1, S1, V1) and subcortical structures (striatum, hippocampus)
\item Number of animals: 8 adult male mice
\item Total recording time: 1,344 hours across all animals
\item Behavioral annotations: 47 distinct behavioral categories automatically classified
\item Environmental context: Enriched arena with food sources, nesting areas, and social zones
\end{itemize}

\subsubsection{Long-term Neural Feature Discovery}

The multi-scale nature of MSAEs proves particularly well-suited for analyzing the diverse temporal dynamics present in long-term naturalistic recordings:

\textbf{Circadian and ultradian features:}
\begin{itemize}
\item \textbf{Day-night activity cycles}: Features capturing 24-hour rhythms in neural activity across brain regions, with phase differences reflecting functional hierarchy
\item \textbf{Sleep-wake transitions}: 6 distinct features encoding rapid state transitions between wake, NREM, and REM sleep states
\item \textbf{Ultradian rhythms}: Features capturing 90-120 minute cycles in activity patterns, particularly prominent in hippocampal recordings
\item \textbf{Meal-related cycles}: Features linked to feeding behavior showing 3-4 hour periodicity aligned with natural foraging patterns
\end{itemize}

\textbf{Behavioral episode features:}
\begin{itemize}
\item \textbf{Foraging sequence patterns}: Features encoding the temporal structure of search, approach, and consumption behaviors during foraging episodes
\item \textbf{Social interaction signatures}: Features specific to periods of social contact, grooming, and territorial behaviors
\item \textbf{Exploration vs. exploitation modes}: Features distinguishing between exploratory behavior in novel environments vs. routine behaviors in familiar areas
\item \textbf{Nesting and rest features}: Features capturing the transition into and maintenance of resting states and nest-building activities
\end{itemize}

\textbf{Learning and adaptation features:}
\begin{itemize}
\item \textbf{Environmental adaptation}: Features that evolve over days/weeks as animals adapt to environmental changes or novel objects
\item \textbf{Plasticity-related patterns}: Features reflecting changes in neural connectivity and response properties over extended periods
\item \textbf{Memory consolidation signatures}: Features active during sleep periods that correlate with behavioral performance improvements
\item \textbf{Habit formation features}: Features tracking the transition from goal-directed to habitual behaviors over weeks of recording
\end{itemize}

\subsubsection{Multi-scale Temporal Analysis}

The Aeon dataset uniquely enables analysis across an unprecedented range of temporal scales:

\textbf{Fast timescales (1ms-1s):}
\begin{itemize}
\item Precise spike timing during specific behaviors (whisking, sniffing, grooming)
\item Coordination between sensorimotor areas during active exploration
\item Fast gamma oscillations (30-80 Hz) during periods of focused attention
\end{itemize}

\textbf{Intermediate timescales (1s-1min):}
\begin{itemize}
\item Behavioral bout structure and transition dynamics
\item Theta oscillations (6-10 Hz) during locomotion and exploration
\item Coordination between brain regions during complex behavioral sequences
\end{itemize}

\textbf{Slow timescales (1min-24h):}
\begin{itemize}
\item Hourly fluctuations in arousal and activity levels
\item Circadian modulation of neural excitability
\item Long-term changes in behavioral preferences and neural responses
\end{itemize}

\subsubsection{Naturalistic Behavior Decoding}

MSAE features enable accurate prediction and classification of complex naturalistic behaviors:

\textbf{Behavioral state classification:}
\begin{itemize}
\item Sleep vs. wake discrimination: 96.2\% accuracy using 10-second windows
\item Active vs. quiet wake states: 91.7\% accuracy
\item REM vs. NREM sleep classification: 88.4\% accuracy
\item Fine-grained behavioral categories: 74.3\% accuracy across 47 behavior types
\end{itemize}

\textbf{Behavioral sequence prediction:}
\begin{itemize}
\item Next behavior prediction (10s horizon): 82.6\% accuracy for coarse categories
\item Behavioral bout duration prediction: Mean absolute error = 3.7 seconds
\item Transition probability estimation: 15\% improvement over Markov chain baselines
\end{itemize}

\textbf{Long-term behavioral forecasting:}
\begin{itemize}
\item Circadian phase prediction: 91.2\% accuracy for next 6-hour period activity levels
\item Sleep onset prediction: 84.7\% accuracy with 30-minute advance warning
\item Feeding behavior prediction: 78.9\% accuracy for next meal timing
\end{itemize}

\subsubsection{Circadian and Homeostatic Dynamics}

The extended recording duration reveals important insights into neural regulation of behavior:

\textbf{Circadian modulation:}
\begin{itemize}
\item Clear day-night differences in feature activation patterns across all brain regions
\item Phase relationships between different brain areas evolve over the circadian cycle
\item Light-dark transitions trigger coordinated changes in neural population dynamics
\end{itemize}

\textbf{Homeostatic regulation:}
\begin{itemize}
\item Sleep pressure accumulation visible in gradually changing feature patterns during wake periods
\item Rebound effects following sleep deprivation captured in altered feature dynamics
\item Evidence for local sleep-like states in cortical areas during prolonged wake periods
\end{itemize}

\textbf{Environmental adaptation:}
\begin{itemize}
\item Novel object introduction triggers reliable changes in exploration-related features
\item Social hierarchy establishment reflected in evolving social interaction features
\item Seasonal changes in temperature and lighting captured in long-term feature evolution
\end{itemize}

% TODO: Add figure for Aeon results
% \begin{figure}[h]
% \centering
% \includegraphics[width=\textwidth]{figures/aeon_features.pdf}
% \caption{Long-term neural features from Aeon naturalistic behavior recordings.}
% \label{fig:aeon_features}
% \end{figure}
