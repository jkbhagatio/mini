\subsection{Additional training details}

\subsubsection{Optimal hyperparameter guidelines}

Based on systematic analysis across datasets, we provide practical guidelines for hyperparameter selection:

\textbf{Latent dimension ($d_{\text{sae}}$):}
\begin{itemize}
\item Small datasets (<1000 neurons): 64-128 latents
\item Medium datasets (1000-5000 neurons): 128-256 latents  
\item Large datasets (>5000 neurons): 256-512 latents
\item Rule of thumb: 0.1-0.2× the number of recorded neurons
\end{itemize}

\textbf{Top-k sparsity:}
\begin{itemize}
\item High interpretability focus: k = 0.05-0.1× latent dimension
\item Balanced interpretability/reconstruction: k = 0.1-0.2× latent dimension
\item High reconstruction focus: k = 0.2-0.3× latent dimension
\end{itemize}

\textbf{Multi-scale configuration:}
\begin{itemize}
\item Minimum scales: Fast (1-10ms), Medium (10-100ms), Slow (100ms-1s)
\item Extended recordings: Add ultra-slow scale (>1s) for circadian and homeostatic dynamics
\item Scale weights: Balance based on analysis goals (equal weights as starting point)
\end{itemize}